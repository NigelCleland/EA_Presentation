%%%%%%%%%%%%%%%%%%%%%%%%%%%%%%%%%%%%%%%%%%%%%%%%%%%%%%%%%%%%
%%  This Beamer template was created by Cameron Bracken.
%%  Anyone can freely use or modify it for any purpose
%%  without attribution.
%%
%%  Last Modified: January 9, 2009
%%

\documentclass[xcolor=x11names,compress]{beamer}

%% General document %%%%%%%%%%%%%%%%%%%%%%%%%%%%%%%%%%
\usepackage{graphicx}
\usepackage{tikz}
\usetikzlibrary{decorations.fractals}
%%%%%%%%%%%%%%%%%%%%%%%%%%%%%%%%%%%%%%%%%%%%%%%%%%%%%%

\usepackage{booktabs}


%% Beamer Layout %%%%%%%%%%%%%%%%%%%%%%%%%%%%%%%%%%
\useoutertheme[subsection=false,shadow]{miniframes}
\useinnertheme{default}
\usefonttheme{serif}
\usepackage{palatino}


\setbeamerfont{title like}{shape=\scshape}
\setbeamerfont{frametitle}{shape=\scshape}

\setbeamercolor*{lower separation line head}{bg=DeepSkyBlue4}
\setbeamercolor*{normal text}{fg=black,bg=white}
\setbeamercolor*{alerted text}{fg=red}
\setbeamercolor*{example text}{fg=black}
\setbeamercolor*{structure}{fg=black}

\setbeamercolor*{palette tertiary}{fg=black,bg=black!10}
\setbeamercolor*{palette quaternary}{fg=black,bg=black!10}

\renewcommand{\(}{\begin{columns}}
\renewcommand{\)}{\end{columns}}
\newcommand{\<}[1]{\begin{column}{#1}}
\renewcommand{\>}{\end{column}}
%%%%%%%%%%%%%%%%%%%%%%%%%%%%%%%%%%%%%%%%%%%%%%%%%%




\begin{document}


%%%%%%%%%%%%%%%%%%%%%%%%%%%%%%%%%%%%%%%%%%%%%%%%%%%%%%
%%%%%%%%%%%%%%%%%%%%%%%%%%%%%%%%%%%%%%%%%%%%%%%%%%%%%%
\section{\scshape Introduction}
\begin{frame}
\title{Reserve in Electricity Markets}
%\subtitle{SUBTITLE}
\author{
    Nigel Cleland\\
    {\it University of Auckland \\
    EPOC}\\
}
\date{

    \vspace{1cm}
    \today
}
\titlepage
\end{frame}

%%%%%%%%%%%%%%%%%%%%%%%%%%%%%%%%%%%%%%%%%%%%%%%%%%%%%%
%%%%%%%%%%%%%%%%%%%%%%%%%%%%%%%%%%%%%%%%%%%%%%%%%%%%%%
\begin{frame}{Introduction}
\tableofcontents
\end{frame}

\begin{frame}{About Me}
\begin{itemize}
\item University of Canterbury, BE(Hons) Chemical and Process Engineering
\item University of Auckland, Year Three, Ph.D Eng. Sci and C\&M
\item Prior work at load aggregators
\item HVDC Pole 3 Commissioning (Trading Team)
\item Based at Transpower S.O. 2013
\item Various Consulting Jobs
\end{itemize}
\end{frame}

\begin{frame}{Rough Agenda}
\begin{itemize}
\item Reserve Constraints
\item Assessment of Spot Prices
\item Equilibrium Models of Reserve Participants
\item Visualising Energy and Reserve Offers
\item Using Bayesian Probability to assess Constraints
\item Theoretical HVDC Transfer Capabilities
\item Open Source and Open Data
\end{itemize}
\end{frame}

%%%%%%%%%%%%%%%%%%%%%%%%%%%%%%%%%%%%%%%%%%%%%%%%%%%%%%
%%%%%%%%%%%%%%%%%%%%%%%%%%%%%%%%%%%%%%%%%%%%%%%%%%%%%%
\section{\scshape Reserve Constraints}
\begin{frame}
\vspace{1.5cm}
\begin{center}
{\Huge\textit{Reserve Constraints}}
\end{center}
\end{frame}

\begin{frame}{\scshape It starts with a picture}
\begin{figure}
%\includegraphics[width=0.95\textwidth]{}
\caption{Haywards Nodal Spot Price (x axis) compared with the North Island
FIR Price (y axis)}
\end{figure}
\end{frame}

%%%%%%%%%%%%%%%%%%%%%%%%%%%%%%%%%%%%%%%%%%%%%%%%%%%%%%
%%%%%%%%%%%%%%%%%%%%%%%%%%%%%%%%%%%%%%%%%%%%%%%%%%%%%%
\begin{frame}{\scshape Why does this matter?}
\begin{figure}
\includegraphics[width=0.95\textwidth]{img/reserveprice.png}
\caption{Revenue ``lost'' for missing highly priced trading periods}
\end{figure}
\end{frame}

\begin{frame}{\scshape Effect on Individual Consumers}
\begin{table}
\caption{Monthly Revenue ``missed'' by various IL producers}
\begin{tabular}{cccc}
\toprule
& NZST & PPAC & SKOG \\
\midrule
2009 & 18-85\% & 2-92\% & 30-80\% \\
2010 & 4-90\% & 0-90\% & 5-70\% \\
\bottomrule
\end{tabular}
\end{table}
In November 2010 NZST missed 90\% of the monthly IR Revenue, SKOG missed 6\%
\vspace{2cm}
\end{frame}

%%%%%%%%%%%%%%%%%%%%%%%%%%%%%%%%%%%%%%%%%%%%%%%%%%%%%%
%%%%%%%%%%%%%%%%%%%%%%%%%%%%%%%%%%%%%%%%%%%%%%%%%%%%%%
\begin{frame}{\scshape Some Theory}
\scalebox{0.75}{
\begin{minipage}{0.50\textwidth}
\begin{eqnarray*}
[POPF] \min & p_g^T g + p_r^T r & \\
\text{st.} &  Mg + Af = d &[\pi] \\
                  &  r + g \le G &[\epsilon] \\
                  &  r - Kg \le 0 &[\kappa]  \\
                  & Er - g \ge 0 &[\lambda^{1}] \\
                  & Hr - Bf \ge 0 &[\lambda^{2}] \\
                  & r \le R &[\omega] \\
                  & |f| \le F &[\tau^{\pm}] \\
                  & Lf  = 0 & [\alpha] \\
                  & r, g \geq 0 & \\
\end{eqnarray*}
\end{minipage}}
\scalebox{0.75}{
\begin{minipage}{0.3\textwidth}
\begin{eqnarray*}
[DOPF] \max & d^{T} + R^{T}\omega + G^{T}\epsilon + F^{T}(\tau^{+} +
\tau^{-}) & \\
\text{st.} & M^{T}\pi + \epsilon - K\kappa + \lambda^{1} \le p_g &[g] \\
                  & \omega + \epsilon + \kappa + E\lambda^{1} \le p_r &[r]  \\
                  & A^{T}\pi + \tau^{+} - \tau^{-} - B^{T}\lambda^{2} +L^{T}\alpha = 0  &[f] \\
                  & \omega, \epsilon, \tau^{\pm}, \kappa \le 0 & \\
                  & \lambda^{1}, \lambda^{2} \ge 0 & \\
                  & & \\
                  & & \\
                  & & \\
                  & & \\
\end{eqnarray*}
\end{minipage}}
\end{frame}

\begin{frame}{\scshape Case Studies}
\begin{figure}
\includegraphics[width=0.95\textwidth]{img/nodal_diagram.pdf}
\caption{Some Case Studies to illustrate different mechanisms of binding
constraints occurring}
\end{figure}
\end{frame}


%%%%%%%%%%%%%%%%%%%%%%%%%%%%%%%%%%%%%%%%%%%%%%%%%%%%%%
%%%%%%%%%%%%%%%%%%%%%%%%%%%%%%%%%%%%%%%%%%%%%%%%%%%%%%
\begin{frame}{\scshape Case Study Results}
Marginal Risk Setting Generator
\begin{equation}
\pi = p_{g,marginal} - \lambda
\end{equation}
Risk Constrained Transmission Line
\begin{equation}
\pi_2 = \pi_1 - \lambda_2
\end{equation}
Bathtub Constrained Transmission
\begin{equation}
\pi_2 = \dfrac{1}{1+k_{g,2}}p_{g,2} + \dfrac{k_{g,2}}{1+k_{g,2}}(\pi_1 + \lambda_{2})
\end{equation}
\end{frame}

\begin{frame}{Testing These, Marginal Generator}
\begin{figure}
\includegraphics[scale=0.3]{img/ccgt_reserve_prices_offer_prices.png}
\caption{Reserve Constraints binding upon major CCGT Units}
\end{figure}
\end{frame}

\begin{frame}{Testing These, Marginal Transmission, NI}
\begin{figure}
\includegraphics[width=0.95\textwidth]{img/ni_reserve_prices_island_differential.png}
\caption{Reserve Constraints Binding upon Northward HVDC Transmission}
\end{figure}
\end{frame}

\begin{frame}{Testing These, Marginal Transmission, SI}
\begin{figure}
\includegraphics[width=0.95\textwidth]{img/si_reserve_prices_island_differential.png}
\caption{Reserve Constraints Binding upon Southward HVDC Transmission}
\end{figure}
\end{frame}

\begin{frame}{Testing These, Bathtub Constraints}
\begin{figure}
\includegraphics[width=0.95\textwidth]{img/mrpl_fan_curve.png}
\caption{Mighty River Fan Curve, TP 19, October 3 2013.}
\end{figure}
\end{frame}

\section{\scshape Spot Market Prices}
\begin{frame}
\vspace{1.5cm}
\begin{center}
{\Huge\textit{Spot Market Prices}}
\end{center}
\end{frame}

\begin{frame}{Scarcity, Constraints or Both?}
\begin{itemize}
\item How do we understand Price?
\item Moving up a merit order stack?
\item High Demand = High Price?
\item Hydrology? Price = f(Inverse Hydro)
\item Constraints?
\end{itemize}
\end{frame}

\begin{frame}{Average Price at Different Demand}
\begin{figure}
\includegraphics[width=0.95\textwidth]{img/price_vs_demand.pdf}
\caption{The higher the demand, the higher the energy price, we're moving up
the stack.}
\end{figure}
\end{frame}

\begin{frame}{Average Price at Different Hydrology}
\begin{figure}
\includegraphics[width=0.95\textwidth]{img/price_and_hydrology.pdf}
\caption{As expected, the less water we have (relative to the lower decile
for the time of year) the higher the average price}
\end{figure}
\end{frame}

\begin{frame}{Average Demand at Different Price Points}
\begin{figure}
\includegraphics[width=0.95\textwidth]{img/demand_and_price.pdf}
\caption{The relationship between high demand and high prices isn't so
clear when the reverse situation occurs}
\end{figure}
\end{frame}

\begin{frame}{Average Hydrology at Different Price Points}
\begin{figure}
\includegraphics[width=0.95\textwidth]{img/hydrology_and_price.pdf}
\caption{The Paradox of Hydrology, the highest price trading periods are
associated with large quantities of water}
\end{figure}
\end{frame}

\begin{frame}{Constraints at different price levels}
\begin{figure}
\includegraphics[width=0.95\textwidth]{img/constrained_periods_analysis.pdf}
\caption{Aggregate assessment of constraints in the New Zealand Market}
\end{figure}
\end{frame}

\begin{frame}{Specific Constraints}
\begin{table}
\caption{Constraints binding during the top 155 priced trading periods}
\begin{tabular}{lrrrr}
\toprule
{} &  Occurences &  Mean &  Min &  Max \\
\midrule
Waikato Block SIR Constraint &          41 &   768 &    0 & 4948 \\
Waikato Block FIR Constraint &          40 &   491 &    2 & 3834 \\
Tokaanu SIR Constraint       &          26 &   417 &    2 & 1010 \\
Waikato Block Dispatch       &          21 &  1409 &   13 & 4653 \\
Tokaanu FIR Constraint       &          13 &  1009 &    0 & 4409 \\
\bottomrule
\end{tabular}
\end{table}
\end{frame}

\begin{frame}{Contextualising the Constraints}
\begin{figure}
\includegraphics[width=0.95\textwidth]{img/gen_mrp_output_vs_baseline.pdf}
\caption{Dispatch (CDF) of Genesis and Mighty River during Constraints Periods
(Genesis for Tokaanu Constraints, Mighty River for Waikato Constraints)
compared with the overall CDF for the providers}
\end{figure}
\end{frame}

%%%%%%%%%%%%%%%%%%%%%%%%%%%%%%%%%%%%%%%%%%%%%%%%%%%%%%
%%%%%%%%%%%%%%%%%%%%%%%%%%%%%%%%%%%%%%%%%%%%%%%%%%%%%%
%%%%%%%%%%%%%%%%%%%%%%%%%%%%%%%%%%%%%%%%%%%%%%%%%%%%%%
%%%%%%%%%%%%%%%%%%%%%%%%%%%%%%%%%%%%%%%%%%%%%%%%%%%%%%
\section{\scshape Results}
\subsection{Frame 1}
\begin{frame}{Frame 1}

\end{frame}

\end{document}
